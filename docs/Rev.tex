\documentclass[a4paper, 12pt]{article}
\usepackage[utf8]{inputenc}
\usepackage[top=2cm, bottom=2cm, left=3cm, right=3cm]{geometry}
\usepackage{amsfonts, amsmath, amssymb}
\usepackage{graphicx}
\usepackage{amsthm}
\usepackage{amsmath}
\usepackage{amsfonts}
\usepackage{amssymb}
\usepackage{ifthen}
\usepackage{fancyhdr}
\usepackage{systeme}
\usepackage{amsthm}
\usepackage{geometry}
\usepackage[brazil]{babel}
\usepackage{indentfirst}
\usepackage{booktabs}
\usepackage{hyperref}
\usepackage[utf8]{inputenc}   
\usepackage{IEEEtrantools}
\usepackage{changepage}
\usepackage{float}
\usepackage{listings}
\usepackage{color}
\usepackage[T1]{fontenc}
\usepackage{xfrac}
\definecolor{dkgreen}{rgb}{0,0.6,0}
\definecolor{gray}{rgb}{0.5,0.5,0.5}
\definecolor{mauve}{rgb}{0.58,0,0.82}
\lstset{
  language=Python,
  literate=
{á}{{\'a}}1
{à}{{\`a}}1
{ã}{{\~a}}1
{é}{{\'e}}1
{ê}{{\^e}}1
{í}{{\'i}}1
{ó}{{\'o}}1
{õ}{{\~o}}1
{ú}{{\'u}}1
{ü}{{\"u}}1
{ç}{{\c{c}}}1,
  basicstyle=\ttfamily\tiny\footnotesize,           
  numbers=left,                   
  numberstyle=\tiny\color{gray},  
  stepnumber=2,                             
  numbersep=5pt,                  
  backgroundcolor=\color{white},    
  showspaces=false,               
  showstringspaces=false,         
  showtabs=false,                 
  frame=single,                   
  rulecolor=\color{black},        
  tabsize=1,                      
  captionpos=b,                   
  breaklines=true,                
  breakatwhitespace=false,        
  title=\lstname,                              
  keywordstyle=\color{blue},          
  commentstyle=\color{dkgreen},       
  stringstyle=\color{mauve}, 
}
\title{Primeiro Trabalho Computacional}
\author{
Victor Rodrigues de Carvalho Leal Monteiro\ 
N°USP: 11809902 \\
\\
{\textbf{Professor: Prof. Dr. Luis Carlos de Castro Santos}}}
\date{IME-USP 2025}
\begin{document}
\maketitle

%==========================================================

\section*{Dedução de \( g(x,t) \)}

Dada a EDP parabólica e a solução manufaturada proposta:
\[
u_t = u_{xx} + g(x,t), \quad 0 < x < 1
\]
\[
u(x,t) = e^{-t} \sin\left( \frac{\pi}{2} x \right) \cos\left( \frac{\pi}{2} x \right)
\]

Usando a identidade trigonométrica \(\sin a \cos a = \frac{1}{2}\sin(2a)\), obtém-se:
\[
u(x,t) = \frac{1}{2} e^{-t} \sin(\pi x)
\]

Derivando em relação a \(t\):
\[
u_t = -\frac{1}{2} e^{-t} \sin(\pi x)
\]

Derivando duas vezes em relação a \(x\):
\[
u_{xx} = -\frac{\pi^2}{2} e^{-t} \sin(\pi x)
\]

Substituindo na EDP e isolando \(g(x,t)\):
\[
-\frac{1}{2} e^{-t} \sin(\pi x) = -\frac{\pi^2}{2} e^{-t} \sin(\pi x) + g(x,t)
\]
\[
\boxed{g(x,t) = \frac{\pi^2 - 1}{2} e^{-t} \sin(\pi x)}
\]

Dado o problema de condições de contorno de Dirichlet resultante, aplicaram-se ambos os métodos Forward Difference e Crank-Nicolson em Python, cujos resultados serão exibidos abaixo.

%==========================================================
\pagebreak

\section{Resultados Numéricos}

\subsection{Tabela de Convergência}

\begin{table}[H]
\centering
\caption{Resultados do método Forward Difference para diferentes refinamentos de malha}
\label{tab:convergencia}
\begin{tabular}{cccccc}
\toprule
$h$ & $k$ & $N_t$ & $k/h^2$ & Erro $L^2$ & Taxa de Convergência \\
\midrule
0.1000 & 0.004926 & 203 & 0.4926 & $1.37 \times 10^{-3}$ & -- \\
0.0500 & 0.001238 & 808 & 0.4952 & $3.42 \times 10^{-4}$ & 2.00 \\
0.0250 & 0.000310 & 3226 & 0.4960 & $8.51 \times 10^{-5}$ & 2.01 \\
\bottomrule
\end{tabular}
\end{table}

\section{Análise Gráfica}

\begin{figure}[H]
\centering
\includegraphics[width=\textwidth]{graficos_dados_reais.png}
\caption{Resultados das simulações numéricas: (a) Convergência do erro, (b) Comparação entre soluções numérica e analítica, (c) Relação entre os passos espacial e temporal}
\label{fig:resultados}
\end{figure}

\subsection{Análise do Gráfico de Convergência}

O gráfico de convergência demonstra o comportamento do erro na norma $L^2$ em função do refinamento da malha espacial $h$. Observa-se que:

\begin{itemize}
    \item O erro decresce quadraticamente com $h$, conforme esperado teoricamente
    \item A taxa de convergência calculada aproxima-se de 2, validando a implementação do método
    \item A linha tracejada vermelha representa a convergência teórica $O(h^2)$
    \item Os pontos azuis representam os erros reais obtidos nas simulações
\end{itemize}

A relação observada confirma que o método Forward Difference possui ordem de convergência $O(h^2)$ quando o passo temporal $k$ é escolhido proporcional a $h^2$.

\subsection{Análise da Comparação de Soluções}

O gráfico de comparação mostra a solução numérica e analítica no tempo final $t = 1$ para $h = 0.1$. Nota-se:

\begin{itemize}
    \item Boa concordância entre as soluções numérica e analítica
    \item Pequenas discrepâncias devido aos erros de truncamento do método
    \item O erro máximo local é indicado no gráfico, proporcionando uma medida da precisão pontual
    \item As condições de contorno são satisfeitas corretamente ($u(0,t) = u(1,t) = 0$)
\end{itemize}

A forma senoidal da solução é preservada pelo método numérico, demonstrando sua adequação para este tipo de problema.

\subsection{Análise da Relação k vs h}

O gráfico da relação entre os passos temporal e espacial ilustra:

\begin{itemize}
    \item Os valores de $k$ utilizados nas simulações (pontos verdes)
    \item O limite teórico de estabilidade $k = \frac{1}{2}h^2$ (linha tracejada vermelha)
    \item A abordagem conservadora adotada, mantendo $k$ ligeiramente abaixo do limite
    \item A relação quadrática entre $k$ e $h$ necessária para estabilidade
\end{itemize}

Esta relação é fundamental para garantir que erros numéricos não amplifiquem-se exponencialmente durante a simulação.

\section{Discussão dos Resultados}

\subsection{Convergência e Precisão}

Os resultados da Tabela \ref{tab:convergencia} confirmam a convergência de segunda ordem do método Forward Difference. A taxa de convergência calculada de aproximadamente 2.0 indica que o método está operando dentro das expectativas teóricas.

\subsection{Eficiência Computacional}

Observa-se que o número de passos temporais $N_t$ aumenta significativamente com o refinamento da malha, seguindo a relação $N_t \propto h^{-2}$. Isto representa uma limitação prática do método explícito para malhas muito refinadas.

\subsection{Estabilidade Numérica}

A seleção conservadora de $k = 0.99 \times \frac{1}{2}h^2$ garantiu estabilidade em todas as simulações, com $k/h^2$ mantendo-se consistentemente abaixo do limite crítico de 0.5.

\section{Conclusão}

A implementação do método Forward Difference demonstrou ser eficaz para a resolução da EDP parabólica em estudo. Os resultados numéricos validam:

\begin{itemize}
    \item A ordem de convergência teórica $O(h^2)$ do método
    \item A importância do critério de estabilidade para simulações bem-sucedidas
    \item A precisão adequada para aplicações práticas
    \item As limitações computacionais inerentes aos métodos explícitos
\end{itemize}

O método mostrou-se robusto e confiável dentro dos parâmetros de estabilidade estabelecidos, fornecendo soluções numéricas consistentes com a solução analítica.

\end{document}
